\documentclass[a4paper,titlepage]{article}
\usepackage[italian]{babel}
\usepackage{graphicx}
\usepackage{float}
\usepackage[utf8]{inputenc}
\usepackage{amsmath}
\usepackage[noend]{algorithmic}
\usepackage{color}
\usepackage{floatpag}
\usepackage{listings}
\usepackage{textcomp}
\usepackage{caption}
\usepackage{afterpage}
\usepackage{titling}
\usepackage{titlesec}
\usepackage{pbox}
\usepackage{tabularx}
\linespread{1.3}
\newcommand{\subtitle}[1]{%
  \posttitle{%
    \par\end{center}
    \begin{center}\large#1\end{center}
    \vskip0.5em}%
}
\renewcommand{\lstlistingname}{Codice}

\definecolor{listinggray}{gray}{0.9}
\definecolor{lbcolor}{rgb}{0.9,0.9,0.9}
\lstset{
backgroundcolor=\color{lbcolor},
		tabsize=3,
		linewidth=13.5cm,
		language=MATLAB,
        basicstyle=\scriptsize,
        upquote=true,
        aboveskip={1.5\baselineskip},
        columns=fixed,
        showstringspaces=false,
        extendedchars=false,
        breaklines=true,
        prebreak = \raisebox{0ex}[0ex][0ex]{\ensuremath{\hookleftarrow}},
        frame=single,
        numbers=left,
        showtabs=false,
        showspaces=false,
        showstringspaces=false,
        identifierstyle=\ttfamily,
        keywordstyle=\color[rgb]{0,0,1},
        commentstyle=\color[rgb]{0.026,0.112,0.095},
        stringstyle=\color[rgb]{0.627,0.126,0.941},
        numberstyle=\color[rgb]{0.205, 0.142, 0.73}
}
\lstset{
    backgroundcolor=\color{lbcolor},
    tabsize=2,
  language=sh,
  captionpos=b,
  tabsize=3,
  frame=lines,
  numbers=left,
  numberstyle=\tiny,
  numbersep=5pt,
  breaklines=true,
  showstringspaces=false,
  basicstyle=\footnotesize,
%  identifierstyle=\color{magenta},
  keywordstyle=\color[rgb]{0,0,1},
  commentstyle=\color{green},
  stringstyle=\color{red}
  }


\titlespacing*{\subsubsection}{0pt}{1.1\baselineskip}{\baselineskip}

\begin{document}
\pagenumbering{gobble}
%topmatter
\title{Assignment 1 - Medical Informatics}
\subtitle{Clinical message parsing}
\author{Gerardo Vitagliano - 2017214620
		Martin Sogenannter - 2017-----}
\date{\vspace{-5ex}}
\vfill
\maketitle
\clearpage
\section{Message generation}

          Il seguente documento analizza le prestazioni della funzione myExp(), definita nello script matlab myExp.m. Tale funzione effettua il calcolo
		  dell'esponenziale di un numero scalare reale con precisione determinata dall'utente.
		  In particolare, si presenteranno i risultati dei casi di test effettuati tramite il framework appositamente previsto da MATLAB.
		  I test case previsti sono volti a verificare in primo luogo l'efficacia della funzione, poi la robustezza nei casi in cui i parametri forniti
		  non rispettino le specifiche richieste, e infine l'accuratezza della funzione e il suo errore relativo.
		  		 L'accuratezza della funzione è stata calcolata considerando l'errore relativo rispetto alla funzione exp() già presente nella libreria default di MATLAB.
		 Per quantificare l'accuratezza si farà uso della formula:\\
			\centerline{$|f(x)-e^x|/f(x)$}
		 Dove con f(x) si è definita la funzione myExp() implementata.\\
		In appendice saranno riportati i codici degli script in \textbf{MATLAB} utilizzati per effettuare il testing.

		\subsection{Casi di test}
		Ciascun test è stato implementato come funzione all'interno del framework per i test case di MATLAB.
		Di seguito si analizzerà il codice di ciascuno di essi e se ne riporterà il risultato.
		I casi di test sono stati progettati ed eseguiti secondo la seguente tabella:


\subsubsection{TC1}
Questo test case vede come input il numero reale scalare positivo 10, e pertanto l'output previsto è il corrispondente esponenziale $e^{10}$.
Si tiene in considerazione un'accuratezza entro limiti di $3*\epsilon$.
\begin{lstlisting}[caption=Test Case 1]
	function testPositiveSolution(testCase)
		actSolution = myExp(10);
		expSolution = exp(10);
		verifyEqual(testCase,actSolution,expSolution,'RelTol', 3*eps);
	end
\end{lstlisting}


\section{Message parsing}

Al lancio della suite dei casi di test tutti i test sono stati superati. L 'output di MATLAB è il seguente:
\begin{lstlisting}[caption=Risultati]
>> results = runtests('expTests.m')
results = 1*16 TestResult array with properties:
    Name
    Passed
    Failed
    Incomplete
    Duration
    Details
Totals:   16 Passed, 0 Failed, 0 Incomplete.
   0.29981 seconds testing time.
\end{lstlisting}

\clearpage

Di seguito si riportano i risultati del confronto tra la funzione implementata (considerata con massima accuratezza) e la funzione primitiva di MATLAB exp().
Il primo grafico mette a confronto i risultati ottenuti tra i due: come si può notare non c'è sostanziale differenza, ma i due grafici si sovrappongono.
Il secondo considera invece l'errore relativo al crescere del parametro x.\\
Nel secondo grafico si  può notare come non esista dipendenza lineare tra errore e grandezza degli ingressi. Tuttavia,
facendo uso dell'interpolazione polinomiale offerta da MATLAB (in rosso) , si osserva un andamento leggermente crescente al crescere del parametro x.

\clearpage
\appendix
\section{Appendix: Script}
\subsection{parser.m}
{\tt  \lstinputlisting{parser.m}}
