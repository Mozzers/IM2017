\documentclass[a4paper,titlepage]{article}
\usepackage[italian]{babel}
\usepackage{graphicx}
\usepackage{float}
\usepackage[utf8]{inputenc}
\usepackage{amsmath}
\usepackage[noend]{algorithmic}
\usepackage{color}
\usepackage{floatpag}
\usepackage{listings}
\usepackage{textcomp}
\usepackage{caption}
\usepackage{afterpage}
\usepackage{titling}
\usepackage{titlesec}
\usepackage{pbox}
\usepackage{tabularx}
\linespread{1.3}
\newcommand{\subtitle}[1]{%
  \posttitle{%
    \par\end{center}
    \begin{center}\large#1\end{center}
    \vskip0.5em}%
}
\renewcommand{\lstlistingname}{Code}

\definecolor{listinggray}{gray}{0.9}
\definecolor{lbcolor}{rgb}{0.9,0.9,0.9}
\lstset{
backgroundcolor=\color{lbcolor},
		tabsize=3,
		linewidth=13.5cm,
		language=MATLAB,
        basicstyle=\scriptsize,
        upquote=true,
        aboveskip={1.5\baselineskip},
        columns=fixed,
        showstringspaces=false,
        extendedchars=false,
        breaklines=true,
        prebreak = \raisebox{0ex}[0ex][0ex]{\ensuremath{\hookleftarrow}},
        frame=single,
        numbers=left,
        showtabs=false,
        showspaces=false,
        showstringspaces=false,
        identifierstyle=\ttfamily,
        keywordstyle=\color[rgb]{0,0,1},
        commentstyle=\color[rgb]{0.026,0.112,0.095},
        stringstyle=\color[rgb]{0.627,0.126,0.941},
        numberstyle=\color[rgb]{0.205, 0.142, 0.73}
}
\lstset{
    backgroundcolor=\color{lbcolor},
    tabsize=2,
  language=sh,
  captionpos=b,
  tabsize=3,
  frame=lines,
  numbers=left,
  numberstyle=\tiny,
  numbersep=5pt,
  breaklines=true,
  showstringspaces=false,
  basicstyle=\footnotesize,
%  identifierstyle=\color{magenta},
  keywordstyle=\color[rgb]{0,0,1},
  commentstyle=\color{green},
  stringstyle=\color{red}
  }


\titlespacing*{\subsubsection}{0pt}{1.1\baselineskip}{\baselineskip}

\begin{document}
\pagenumbering{gobble}
%topmatter
\title{Assignment 2 IM - ECG Analysis}
\author{Gerardo Vitagliano - 2017214620 \\ Martin Schlegel - 2017190694 }
\date{\vspace{-5ex}}
\vfill
\maketitle
\clearpage
\section{Classification System}

\subsection{PVC}
For the PVC detection we use differences in the RR regularitys, the QRS areas and the hermite functions.
The detection via the RR regularitys and the QRS areas give a decimal value to detect the specific amount of differences. The hermite functions just can decide between true and false values. So the resulting rule and thresholds are following:\\\\
$myPVCRR(i) >= 1 || myPVCArea(i) >= 1 || (myPVCArea(i) + myPVCRR(i) >= 1.9) \&\& myPVCHermit(i) == 1$
\\\\
If the RR regularitys or the QRS areas too different we assume a PVC. If both of them quite sure and the hermite is true, than we assume a PVC as well.

\subsection{AF}
For the AF detection we consider the difference of the RR intervals and the frequency for different areas of the ecg.\\\\
$sdnnRRWindows(i) < 0.9 \&\& lfhfWindows(i) < 0.4$
\\\\
blabla TODO

\section{Methods}

\subsection{PVC}
- RR regularitys: just following the slides, take the indexes of the peaks put them in the formula
- QRS area: we take the 0.06 seconds around a peak and calculate these are via $trapz()$. That value we calculate via the given formulas on the slides as well.
- hermite: ??

\subsection{AF}
- RR intervals: same as for PVC just for all the different windows, than we compare the window values and try to get smart assumptions
- frequency: just folllowing the slides -> output is shit TODO ??

\section{Results}

\subsection{PVC}
TODO Gerardo

\subsection{AF}
TODO Gerardo

\section{Discussion}

\subsection{PVC}
MUITO BEM!!!

\subsection{AF}
0.5 * muito bem!

\end{document}
